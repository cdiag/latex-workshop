\documentclass[a4paper,10pt]{article}
\usepackage[utf8]{inputenc}
\usepackage{amsmath}
\usepackage{amssymb}


\begin{document}
Hinweis: Lösungen finden sich auf Wikipedia - es geht nur um die Darstellung der Formel in LaTeX!
\section{Wichtige Naturkonstanten}
%Bitte in wissenschaftlicher Notation mit Einheiten
\begin{center}
\begin{tabular}{ll}
Elementarladung e & -  \\
Hubble-Konstante $H_0$  & - 
\end{tabular}
\end{center}

\section{Schwarzschildradius eines nicht rotierenden schwarzen Loches}

\section{Das Kommutativgesetz der Aussagenlogik}


\section{Fourier}
Sei $f\in L^{1}(\mathbb{R}^{n})$ eine integrierbare Funktion. Die (kontinuierliche) Fourier-Transformierte $\mathcal{F}f$ von $f$ ist definiert durch:\footnote{Quelle: https://de.wikipedia.org/wiki/Fourier-Transformation}:


\end{document}


