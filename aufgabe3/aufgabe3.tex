\documentclass[a4paper,10pt]{article}
\usepackage[utf8]{inputenc}
\usepackage{amsmath}
\usepackage{amssymb}


\begin{document}
Hinweis: Lösungen finden sich auf Wikipedia - es geht nur um die Darstellung der Formel in LaTeX!
\section{Wichtige Naturkonstanten}
%Bitte in wissenschaftlicher Notation mit Einheiten
\begin{center}
\begin{tabular}{ll}
Elementarladung e & $1,602 \cdot 10^{-19}J$  \\
Hubble-Konstante $H_0$  & $70,5\frac{\text{km}}{\text{s}\cdot \text{Mpc}}$ 
\end{tabular}
\end{center}

\section{Schwarzschildradius eines nicht rotierenden schwarzen Loches}
\begin{equation}
  R = \frac{2GM}{c^2}
\end{equation}

\section{Das Kommutativgesetz der Aussagenlogik}

\begin{equation*}
  A \vee B \iff B \vee A
\end{equation*}

\begin{equation*}
  A \land B \iff B \land A
\end{equation*}


\section{Fourier}
Sei $f\in L^{1}(\mathbb{R}^{n})$ eine integrierbare Funktion. Die (kontinuierliche) Fourier-Transformierte $\mathcal{F}f$ von $f$ ist definiert durch:\footnote{Quelle: https://de.wikipedia.org/wiki/Fourier-Transformation}:

\begin{equation}
  (\mathcal{F}f)(y) = \frac{1}{\sqrt{2\pi}^n} \int_{\mathbb{R}^n}f(x)e^{-iy\cdot x}dx
\end{equation}


\end{document}


