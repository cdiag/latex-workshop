\documentclass[a4paper,12pt]{article}
\usepackage[utf8]{inputenc}
\usepackage{amsmath}

\begin{document}

\section{Auswertung}

\begin{table} 
  \begin{tabular}{l|l|l}    
    Spannung [V]  & Widerstand [$\Omega$] & gemessener Strom [A] \\ \hline
    1 & 0.1 & 10.10 \\
    1.4 & 0.1 & 13.84 \\
    1.9 & 0.1 & 18.74 \\ 
    1 & 10 & 0.11 \\
    1.5 & 10 & 0.14 \\
    1.8 & 10 & 0.18 \\ 
  \end{tabular}
  
 \caption{Messwerte des ersten Versuchs}
 \label{table:messwerte}
\end{table}

Wie wir deutlich herausarbeiten konnten, ist das Ohmsche Gesetz gültig. Tabelle \ref{table:messwerte} zeigt einen eindeutigen linearen Zusammenhang zwischen Spannung, Widerstand und Stromfluss. 

\section{Anhang}

\listoftables

\end{document}





