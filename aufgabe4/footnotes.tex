\documentclass[a4paper,10pt]{article}
\usepackage[utf8]{inputenc}
\usepackage{amsmath}


\begin{document}
\section{Erwartung}
Georg Simon Ohm formulierte das nach ihm benannte Ohmsche Gesetz zu:
 $$R = \frac{U}{I} = const$$
Umstellen nach I ergibt leicht:
 $$I = \frac{U}{R}$$

\section{Fehlerabschätzung}

Folgende Eingangsfehler sind bekannt:
\begin{itemize}
 \item Genauigkeit des Voltmeters: 1\%
 \item Genauigkeit des Referenzwiderstandes: 0.5\%
 \item Genauigkeit des Amperemeters: 1\%
\end{itemize}

Es gilt die Gaussche Fehlerfortpflanzung mit

\begin{equation}
 u = \sqrt{\sum_{i=1}^{n}\left[ \frac{\partial f(x)}{\partial x_i} \right]^2 \cdot u_{x_i}^2}
\end{equation}
Damit ergibt sich für das Ohmsche Gesetz: 
\begin{align}
 u &= \sqrt{\left( \frac{\partial I}{\partial U}u_U \right)^2 + \left( \frac{\partial I}{\partial R}u_R \right)^2}
\end{align}
\begin{align}
 &= \sqrt{\left(R^{-1}u_U\right)^2 + \left( U\cdot R^{-2}u_R \right)^2}
 \label{eq:gauss}
\end{align}
Setzt man die Werte aus den Versuchsreihen nun in (\ref{eq:gauss}) ein, so erhält man...

\end{document}


